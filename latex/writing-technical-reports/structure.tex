\section{Structure}

\label{sec:structure}

Beyond the front matter and appendices, your report should contain the
following sections (in order): Introduction, Background, Analysis,
Design, Implementation, User's Guide and/or Examples, Evaluation, and
Conclusion. On occasion, you can omit certain sections, depending on
your programming assignment text. These sections, among other details,
are covered in-depth in the following subsections.

\subsection{Front Matter and Paging}

The front page of your report should present the reader with a title,
list the course / activity and affiliation, list the authors, and
state the final date of your report. You may also add author contact
information, document and/or software version information, etc.,
depending on the context and requirements.

The pages of your report should be numbered, and the headers and/or footers
should provide some contextual information. This is so that if pages are ripped
out of context, they can be stitched back together, in order.

\subsection{Introductory Matters}

A technical report should begin with an abstract, stating the purpose,
and scope of your report. The purpose is usually to present a piece of
software that solves a particular problem --- state this problem
\emph{clearly}, and clearly state the extent of your solution.

It can be a good idea to write an abstract up-front, and to revise it
right before submitting your report. An abstract can help you stay on
track as you write your report, and will tell the reader whether
reading the report is worth their time and effort.

The abstract can be followed by a longer introduction, motivating and
explaining the problem further. However, depending on your task, the
problem may already have been clearly presented to you, and there is
not much leeway in how to interpret the given task. In this case, a
longer introduction is likely to be unnecessary.

\medskip

A short report may well begin just there. There is little need for a
``table of contents'' for a 1--5-page report. For a longer report, the
reader may want to quickly skip to the parts that they find
interesting, and so a table of contents is in order. In either case,
it can be a good idea to give an overview of your report to conclude
the introduction.

The introduction may also briefly list acknowledgements, if others,
beyond the listed authors, have helped along the way.

\input{\dirpath background}

\input{\dirpath analysis}

\input{\dirpath design}

\input{\dirpath implementation}

\input{\dirpath users-guide-and-examples}

\input{\dirpath evaluation}

\input{\dirpath conclusion}

\input{\dirpath appendices}
