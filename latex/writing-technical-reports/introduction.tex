This document contains suggestions on how to structure a technical
report describing the implementation of a small programming project.
This document \emph{does not} intend to cover how to structure a
technical report for a large software development project, beyond the
scope of a programming course.

\medskip

The software you write, in a programming course, and beyond, will
typically address a particular problem. The job of a technical report
is to give an overview of your solution, explain the structure of your
program, and tell the reader how to work with it. Your report is an
exhibit of in-how-far you have understood, and solved the given
problem. Technical reporting is an indispensable part of
\emph{sustainable} software development.

\medskip

This guide is heavily based on the many guides that came before
it\cite{sestoft2002, foerstehjaelp2005, julegave1996, kunst2007}. We
encourage you to read these at your leisure. As with other guides,
this guide is a social construct.  It may be imprecise, and you may
deviate when you deem it necessary.  However, please respect that this
guide is based on the hard-earned experience of those who wrote
software before us.

This guide was originally prepared for the course in Software
Development 2017 at the Department of Computer Science at the
University of Copenhagen (DIKU). It was then revamped to be more
general, during the course on Distributed Objects 2018 at the
Department of Computer Science at the University of Oslo (IFI/UiO). It
is applicable in other programming courses going forth.
