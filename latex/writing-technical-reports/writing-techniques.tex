\section{Writing Techniques}

\label{sec:writing-techniques}

Your report is an exhibit of in-how-far you have understood, and solved the
given problem. It should give an overview that your implementation alone cannot
provide. As such, your report should be well-formed and easy to read.

The following expands on the items originally listed at the end of
\cite{sestoft2002}:

\begin{itemize}

\item Everything in your report should serve a purpose, don't fill the report
with superfluous or redundant content.

\item Have a target audience in mind. It can be a good idea to state the target
audience of your report in the front, or introductory matter. Some sections may
also target a more technical audience, interested in extending, or integrating
your program, while others may target the end-users.

\item Keep it crisp and clear: Avoid unnecessary ``colouring'' words; avoid
pseudo–scientific formulations; avoid variations for the sake of variations;
describe related ideas with related terms and formulations.

\item Write well: slang, bad grammar, bad spelling, unexplained abbreviations
and notions (relative to the target audience), all impede communication.

\item Give an overview of your report, and make your report easy to navigate.
Number your pages, sections, subsections, figures, etc. Make a table of
contents, if your report is sufficiently long.

\item Use proper academic citation (e.g., as in this report).

\end{itemize}
